%\myChapterStar{Titre}{Titre court}{Ajouter � la table des mati�res? (false|true|chapter|section|subsection|subsubsection -chapter par d�faut-)}
\myChapterStar{Introduction}{}{true}
%\myMinitoc{Profondeur de la minitoc (section|subsection|subsubsection)}{Titre de la minitoc}
\myMiniToc{}{Contents}

%\mySectionStar{Titre}{Titre court}{Ajouter � la table des mati�res? (false|true|chapter|section|subsection|subsubsection -section par d�faut-)}
\mySectionStar{Le contexte du travail}{}{true}

  Avec le développement de l'informatique et d'Internet, les données sont de plus en plus stockées dans des serveurs distants afin de faciliter leur utilisation et de réduire les coûts du matériel nécessaire pour leur stockage. Cependant, avec internet, la sécurité des données n'est pas sans faille. Ainsi il est possible d'accéder de façon frauduleuse à des supports de stockage contenant des données venant de divers individus (personne ou entreprises). Au fil des années nous avons assisté à de nombreuses attaques telles que : les pertes de données, l'hameçonnage, le cheval de Troie. Par exemple, Le 18 février 2021, le California Department of Motor Vehicles (DMV) a alerté les conducteurs californiens qu'ils avaient été victimes d'une fuite de données après que son prestataire de gestion de facturation, Automatic Funds Transfer Services, ait subi une attaque de ransomware. Dans l'optique de trouver une solution à ces problèmes, plusieurs approches telles que le contrôle d'accès ont été proposées. Le contrôle d'accès  permet de déterminer les utilisateurs ou les programmes autorisées à accéder et/ou à modifier des données sécurisées dans le système. Traditionnellement, le contrôle d'accès (CA) est basé soit, sur l'identité d'un utilisateur demandant l'exécution d'une autorisation pour effectuer une opération (par exemple, lire) sur un objet (par exemple, un fichier), soit directement, ou encore par l'intermédiaire de types d'attribut prédéfinis tels que des rôles ou des groupes assignés à cet utilisateur. Les praticiens ont noté que cette approche de contrôle d'accès est souvent lourde à gérer étant donné la nécessité d'associer les autorisations directement aux utilisateurs ou à leurs rôles ou groupes. En outre, les qualificatifs de l'utilisateur : identité, groupes et rôles sont souvent insuffisants pour exprimer les politiques de contrôle d'accès du monde réel. Une alternative consiste à accorder ou à refuser les requêtes des utilisateurs sur la base d'attributs arbitraires de l'utilisateur et d'attributs sélectionnés de l'objet, ainsi que de conditions d'environnement qui pourrait être reconnues globalement et plus pertinentes pour les politiques en cours. Cette approche est souvent appelée contrôle d'accès par attributs (ABAC) \cite{Vin15}.
\paragraph{} ABAC définit un paradigme de contrôle d'accès selon lequel les droits d'accès sont accordés aux utilisateurs grâce à l'utilisation des règles combinant les attributs \cite{Vin15}. Un attribut représente une information élémentaire qui caractérise un utilisateur, un objet, une action ou un environnement d'accès. Ce modèle permet une élaboration des politiques granulaires mais flexibles. En effet, la limite de l'élaboration des politiques réside essentiellement dans les attributs dont il faut tenir compte et dans les conditions que le langage informatique peut exprimer. ABAC permet au plus grand nombre de sujets d'accéder à la plus grande quantité de ressources sans obliger les administrateurs à spécifier les relations entre chaque sujet et objet. Toutefois, le fait qu'ABAC nécessite un fort besoin de provisioning et de maintenance des attributs fait de ce dernier un modèle difficile à administrer. Dans des situations d'urgence, ABAC peut refusé l'accès à une utilisateur légitime suite à l'absence d'une valeur d'attribut lors d'une décision d'accès. Un autre inconvénient d'ABAC est qu'il octroie des pouvoirs absolus à l'administrateur du système. Ainsi dans un système de gestion d'un centre hospitalier l'administrateur peut modifier le résultat d'examen de glycémie d'un patient en indiquant que cet examen est négatif pourtant  il ne l'est pas, et l'infirmière qui est en charge d'administrer une perfusion à ce patient afin de baisser sa fièvre lui administre une perfusion contenant du glucose conformément au résultat de son examen. Ce qui entraine par conséquent la mort du patient. ainsi nous remarquons que, l'administrateur d'un système peut manipuler les données du système de façon frauduleuse et en toute liberté. C'est la raison pour laquelle un nouveau modèle a vu le jour : Il s'agit du modèle HOr -BAC qui consiste à gérer le contrôle d'accès et à surveiller toutes les opérations de bases dans un système d'information.
\paragraph{} HOr-BAC se base sur la structure organisationnelle d'une organisation afin de permettre la spécification des politiques de sécurité contextuelle relative aux permissions. Dans HOr-BAC, les permissions sont attribuées aux unités organisationnelles et les employés affectés à ces unités obtiennent les permissions qui les sont attribuées. Ce modèle permet de gérer le contrôle d'accès et de surveiller toutes les opérations de bases dans un système d'information, empêchant ainsi la création des entités virtuelles au sein du système. HOr-BAC permet de représenter la relation hiérarchique qui existe entre les différentes unités organisationnelles d'une organisation. Par exemple dans une entreprise l'unité organisationnelle RH (Ressources Humaines) est subordonnée par l'unité organisationnelle direction générale. En plus de cela il introduit la notion de parapheur électronique qui est un processus de traitement automatique sécurisé permettant de contrôler et de valider les activités du personnel métier y compris le super-utilisateur conformément à la structure organisationnelle de l'entreprise \cite{theseBenoit}. 



\mySectionStar{La problématique étudiée}{}{true}

 Bien que HOr-BAC s'appuie sur la structure organisationnelle d'une entreprise et empêche la création d'entités virtuelles dans un système d'information, cependant il a des lacunes. En effet, dans HOr-BAC, les politiques de contrôle ne peuvent être définies que sur la base de l'unité organisationnelle à laquelle appartient un employé, ce qui limite ainsi la flexibilité du contrôle d'accès. Ceci est  fréquemment rencontré  dans le domaine du Cloud Computing où on assiste à une explosion des unités organisationnelles et de leurs autorisations lorsqu'on essaie de définir des politiques de contrôle d'accès qui se basent sur des caractéristiques d'entités autres que celles prédéfinies dans HOr-BAC. Par conséquent il ne permet que de définir des politiques de contrôle d'accès à gros grain, c'est-à-dire des politiques qui ne sont définies que sur la base des unités organisationnelles. Par exemple dans HOr-BAC, il est difficile d'exprimer la politique de contrôle d'accès suivante : « dans un système de gestion d'un centre hospitalier l'agent d'assurance d'un patient ne peut voir que les informations relatives à la facture des soins reçus par son patient dans son dossier médicale. Il  n'a pas besoin de connaitre des informations relatives à son état de santé »
\paragraph{} Ainsi, le problème qui se pose est le suivant : comment réaliser l'approche orientée attribut du modèle HOr-BAC afin qu'il permette un contrôle d'accès fin et flexible ?


\mySectionStar{Objectif}{}{true}

L'objectif principal de notre travail est de proposer un modèle de contrôle d'accès permettant un contrôle d'accès à grain fin et flexible. En effet nous partons de l'hypothèse selon laquelle en combinant une spécification de politique flexible et fine granulaire  et une capacité de prise de décision dynamique  qui font la renommée d'ABAC avec la facilité d'administration et le contrôle des actions effectuées par un individu (administrateur ou employé) au sein d'un système d'information, ce qui fait la renommée de HOr-BAC nous obtiendrons un modèle de contrôle d'accès flexible et fine granulaire.   \paragraph{}Comme objectifs secondaires nous avons :
\\
- Notre modèle devra permettre une prise de décision dynamique  
\\
- Il doit être applicable  aux données Cloud
\\
- Il doit obéir à la structure organisationnelle d'une organisation et aux relations hiérarchiques qui existe entre les différentes unités organisationnelles d'une entreprise.
\\
- Il doit permettre de surveiller les différentes opérations effectuées au sein d'un système d'information.


\mySectionStar{Plan de notre travail}{}{true}



\myCleanStarChapterEnd
