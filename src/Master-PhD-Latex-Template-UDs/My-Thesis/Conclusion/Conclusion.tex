%\myChapterStar{Titre}{Titre court}{Ajouter � la table des mati�res? (false|true|chapter|section|subsection|subsubsection -chapter par d�faut-)}
\myChapterStar{Conclusion générale}{}{true}
\myMiniToc{}{Contents}

\mySectionStar{Bilan}{}{true}
Dans ce mémoire, nous nous sommes intéressés à résoudre le problème de flexibilité et de granularité fine des politiques de sécurité du contrôle d'accès Basé sur la hiérarchie organisationnelle.\\
\hspace*{0.5cm} La résolution de ce problème nous a permis dans un premier temps de faire un état de l'art sur les modèles de contrôle d'accès existants. Il en ressort de cette revue de littérature que seul le modèle de contrôle d'accès HOr-BAC permet de contrôler le DBA au sein d'une organisation. Mais le fait que ce modèle exprime des politiques d'accès sur la base de l'unité organisationnelle à laquelle appartient un employé limite la flexibilité de ce modèle. par conséquent il ne permet qu'un contrôle à grain fin des ressources du système. Nous avons ensuite présenté une approche orienté attributs de ce modèle afin une spécification des politiques de sécurité qui permettent un contrôle à grain fin des ressources et ne sont limité que pas la puissance de calcul du langage et la taille des attributs des composants du système. Nous avons terminé ce travail par la proposition d'un langage basé sur le langage XACML qui nous permet de spécifier des politiques de contrôle d'accès basé non seulement sur l'unité organisationnelle à laquelle appartient un employé mais aussi sur les attributs. 


\mySectionStar{Quelques perspectives}{}{true}
Les résultats de notre travail sont encourageants, mais pour conclure ils faut déployer le modèle AHOr-BAC et son langage en grandeur nature sur un système existant .



\myCleanStarChapterEnd
