\myChapter{Un autre exemple complet de fusion consensuelle}{Un autre exemple complet de fusion consensuelle}\label{annexeFusions}
\mySaveMarks
Dans cette annexe...

\mySectionStar{Les schémas des règles de transition}{}{false}
Rappelons que les schémas des transitions (complétés pour prendre en compte les documents non clos) de l'automate permettant de représenter les expansions des répliques partielles suivant la vue $\mathcal{V}_1 = \{A,B\}$ lorsqu'on associe les symboles de Dyck '(' et ')' (resp. '[' et ']') au symbole visible $A$ (resp. $B$) et qu'on associe les symboles '($_{\omega}$' et ')$_{\omega}$' (resp. '[$_{\omega}$' et ']$_{\omega}$') au bourgeon $A_{\omega}$ (resp. $B_{\omega}$) de type $A$ (resp. $B$), sont les suivants:

\begin{table}
	\caption{Les schémas des régles de transition pour notre exemple}
	\begin{flushleft}
	\begin{tabular}[t]{lcp{5.3cm}l}
	$\langle A,w_{1} \rangle$ & $\longrightarrow$ & $(P_{1}, [\langle C,u \rangle, \langle B,v \rangle])$ & si $w_{1}=u[v]$ \\
	
	\end{tabular}
	\begin{tabular}[t]{lcp{5.3cm}l}
	
	$\langle A,w_{2} \rangle$ & $\longrightarrow$ & $(P_{1}, [\langle C,u \rangle, \langle B,w_{11} \rangle])$ & si $w_{2}=uw_{11}$ avec $w_{11}=[_{\omega}\, ]_{\omega}$ \\
	
	\end{tabular}
	\begin{tabular}[t]{lcp{5.3cm}l}
	
	$\langle A,w_{3} \rangle$ & $\longrightarrow$ & $(P_{2}, [\,])$ & si $w_{3}=\epsilon$\\
	
	\end{tabular}
	\begin{tabular}[t]{lcp{5.3cm}l}
	
	$\langle A,w_{4} \rangle$ & $\longrightarrow$ & $(A_{\omega}, [\,])$ & si $w_{4}=(_{\omega}\, )_{\omega}$\\
	
	\end{tabular}
	\begin{tabular}[t]{lcp{5.3cm}l}
	
	$\langle B,w_{5} \rangle$ & $\longrightarrow$ & $(P_{3}, [\langle C,u \rangle, \langle A,v \rangle])$ & si $w_{5}=u(v)$ \\
	
	\end{tabular}
	\begin{tabular}[t]{lcp{5.3cm}l}
	
	$\langle B,w_{6} \rangle$ & $\longrightarrow$ & $(P_{3}, [\langle C,u \rangle, \langle A,w_{4} \rangle])$ & si $w_{6}=uw_{4}$ \\
	
	\end{tabular}
	\begin{tabular}[t]{lcp{5.3cm}l}
	
	$\langle B,w_{7} \rangle$ & $\longrightarrow$ & $(P_{4}, [\langle B,u \rangle, \langle B,v \rangle])$ & si $w_{7}=[u][v]$\\
	
	\end{tabular}
	\begin{tabular}[t]{lcp{5.3cm}l}
	
	$\langle B,w_{8} \rangle$ & $\longrightarrow$ & $(P_{4}, [\langle B,w_{11} \rangle, \langle B,v \rangle])$ & si $w_{8}=w_{11}[v]$\\
	
	\end{tabular}
	\begin{tabular}[t]{lcp{5.3cm}l}
	
	$\langle B,w_{9} \rangle$ & $\longrightarrow$ & $(P_{4}, [\langle B,u \rangle, \langle B,w_{11} \rangle])$ & si $w_{9}=[u]w_{11}$\\
	
	\end{tabular}
	\begin{tabular}[t]{lcp{5.3cm}l}
	
	$\langle B,w_{10} \rangle$ & $\longrightarrow$ & $(P_{4}, [\langle B,w_{11} \rangle, \langle B,w_{11} \rangle])$ & si $w_{10}=w_{11}w_{11}$\\
	
	\end{tabular}
	\begin{tabular}[t]{lcp{5.3cm}l}
	
	$\langle B,w_{11} \rangle$ & $\longrightarrow$ & $(B_{\omega}, [\,])$ & si $w_{11}=[_{\omega}\, ]_{\omega}$\\
	
	\end{tabular}
	\begin{tabular}[t]{lcp{5.3cm}l}
	
	$\langle C,w_{12} \rangle$ & $\longrightarrow$ & $(P_{5}, [\langle A,u \rangle, \langle C,v \rangle])$ & si $w_{12}=(u)v$ \\
	
	\end{tabular}
	\begin{tabular}[t]{lcp{5.3cm}l}
	
	$\langle C,w_{13} \rangle$ & $\longrightarrow$ & $(P_{5}, [\langle A,w_{4} \rangle, \langle C,v \rangle])$ & si $w_{13}=w_{4}v$ \\
	
	\end{tabular}
	\begin{tabular}[t]{lcp{5.3cm}l}
	
	$\langle C,w_{14} \rangle$ & $\longrightarrow$ & $(P_{6}, [\langle C,u \rangle, \langle C,v \rangle])$ & si $w_{14}=uv\neq\epsilon$\\
	
	\end{tabular}
	\begin{tabular}[t]{lcp{5.3cm}l}
	
	$\langle C,w_{15} \rangle$ & $\longrightarrow$ & $(C_\omega,[\,])$ & si $w_{15}=\epsilon$\\
	
	\end{tabular}
	\end{flushleft}
\end{table}

De même...



\myRestoreMarks
